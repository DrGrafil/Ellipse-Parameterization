\documentclass{article}
\usepackage[margin=1in]{geometry}
\usepackage{amsmath}
\usepackage{graphicx}
\usepackage{float}
\usepackage{bm}
%\pagestyle{empty}

\title{\vspace{-2.5cm}Tabulate Equations of Common Ellipse Parameters}
\author{Dr. Elliot Grafil}

\begin{document}
\maketitle
%\thispagestyle{empty}
\section*{Introduction}
This document tabulates the equations needed to deduce any of the seven common parameters of an ellipse given two of its parameters. 
\begin{figure}[H]
\begin{center}
\noindent\makebox[\textwidth]{
  \includegraphics[scale=1.25]{./EllipseDiagram/EllipseDiagram.pdf}
  }
  \caption{Common parameters labeled on an example ellipse. Eccentricity, $e$, is not depicted.}
  \label{fig:boat1}
  \end{center}
\end{figure}
\subsection*{\centering{Parameters}}
\begin{description}
\item[$\boldsymbol{a}$] Semi-Major Axis. The length from the center of the ellipse to the farthest point on the curve.
\item[$\boldsymbol{b}$] Semi-Minor Axis. The length from the center of the ellipse to the nearest point on the curve.
\item[$\boldsymbol{c}$] Linear eccentricity. The length from the center of the ellipse to one of its foci.
\item[$\boldsymbol{d}$] Directrix (distance). The distance along the major axis from the center of the ellipse at which the directrix line lies. Sometimes denoted as $x$ or $y$. In special cases it is the equation for the directrix line.
\item[$\boldsymbol{e}$] Eccentricity. Measurement of deviation from being circular. Sometimes denoted as $\epsilon$. Can be confuse with flattening that shares the symbol $\epsilon$.
\item[$\boldsymbol{\ell}$] Semi-Latus Rectum. The length of a line segment that begins at the focus and makes contact with the ellipse. It is perpendicular to the major axis.
\item[$\boldsymbol{p}$] Focal parameter. The length from one of the two foci to the nearest directrix.
\end{description}

\section*{Other Useful Relations \& Terminology}
\begin{description}
\item[Major Axis] Double the length of the semi-major axis ($2a$). The length of the ellipse at its widest point.
\item[Minor Axis] Double the length of the semi-minor axis ($2b$). The length of the ellipse at its thinnest point.
\item[Focal Length] Double the length of the linear eccentricity ($2e$). The length between the ellipse's two foci.
\item[Flattening] A rarer type of measurement for the deviation from being circular. Flattening is given usually in terms of $a$ and $b$ as $f=\frac{a-b}{a}$ or $e$ as $f=1-\sqrt{1-e^2}$. Sometimes denoted as $\epsilon$.
\item[Latus Rectum] Double the length of the semi-latus rectum ($2\ell$). The chord that passes through a focus and is perpendicular to the major axis.
\end{description}

\section*{How To Use}
For the parameter of interest go to the page labeled with the parameters name. 
The first row and column are labeled with different variables. 
Select a row and column based on what information you already have. 
The equation that is displayed in the intersection is the function used to derived the parameter given the two variables. 

\section*{$\dagger$ Note On Directrix, $d$, and Semi-Latus Rectum, $\ell$, Equations}
To solve for the parameters of an ellipse in terms of the Semi-Latus Rectum, $\ell$, and Directrix, $d$ require solving a cubic function in all cases. For example in the case of linear eccentricity, $c$, the roots of the following equation needs to be found:
\begin{align*} % the "starred" equation environments produce no equation numbers
c^3-2c^2 d+c d^2-d \ell^2 = 0 
\end{align*}
Solving using the normal general solution to the cubic equation gives a piecewise solution that is daunting to calculate. The results would span longer than the page in some cases.
Using a trigonometric cubic solution provides a much more manageable solution. However it gives a piecewise solution:
\begin{align*} 
c =\begin{cases}
 &  \dfrac{2d}{3}  \left (1 + \text{sin} \left (\frac{\text{arcsin}\left (1 - \frac{27 \ell^2}{2 d^2}\right )}{3} \right ) \right )\\ 
 & \dfrac{4d}{3}  \text{sin}^2 \left (\frac{\text{arccos}\left (1 - \frac{27 \ell^2}{2 d^2}\right )}{6} \right )
\end{cases}   
\end{align*}
While much more simplified than the general solution, the domain when given as a function of both $d$ and $\ell$ is troublesome to calculate since the switch occurs at $\frac{a}{b}=\sqrt{\frac{2}{3}}$ which when put in terms of $d$ and $\ell$ is ugly.
\newpage

\section*{\centering{Semi-Major Axis}}
\begin{center}
\noindent\makebox[\textwidth]{%
\includegraphics[scale=.95]{./AEquation/AEllipseEquations.pdf}
}
\newpage

\section*{\centering{Semi-Minor Axis}}
\noindent\makebox[\textwidth]{%
\includegraphics[scale=.95]{./BEquation/BEllipseEquations.pdf}
}
\newpage

\section*{\centering{Linear Eccentricity}}
\noindent\makebox[\textwidth]{%
\includegraphics[scale=.95]{./CEquation/CEllipseEquations.pdf}
}
\newpage

\section*{\centering{Directrix}}
\noindent\makebox[\textwidth]{%
\includegraphics[scale=.95]{./DEquation/DEllipseEquations.pdf}
}
\end{center}
\newpage

\section*{\centering{Eccentricity}}
\noindent\makebox[\textwidth]{%
\includegraphics[scale=.95]{./EEquation/EEllipseEquations.pdf}
}
\newpage

\section*{\centering{Semi-Latus Rectum}}
\noindent\makebox[\textwidth]{%
\includegraphics[scale=.95]{./LEquation/LEllipseEquations.pdf}
}
\newpage

\section*{\centering{Focal Parameter}}
\noindent\makebox[\textwidth]{%
\includegraphics[scale=.95]{./PEquation/PEllipseEquations.pdf}
}
\newpage

\end{document}